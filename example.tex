\documentclass{oxblue-beamer}

\title[Sample Presentation]{
    Sample Presentation with \texttt{oxblue-beamer} Class
}
\author[Dela Cruz, Mendoza, et al.]{
    Abby Dela Cruz\inst{1} \and
    Ben Mendoza\inst{2} \and
    Crisostomo Ibarra\inst{2}
}
\institute{
    \inst{1}
    College of Science\\
    University of the Philippines, Diliman \and

    \inst{2
    }Ateneo Center for Economic Research and Development\\
    Ateneo de Manila University \and
}
\date{01 January 2000}

\begin{document}

\begin{frame}
\titlepage
\end{frame}

% Table of Contents
\begin{frame}{Table of Contents}
\tableofcontents
\end{frame}

\section{Introduction}

\begin{frame}{Introduction}
    The \texttt{oxblue-beamer} class is a simple and clean Beamer theme that is suitable for academic presentations.
    \bigskip
    \begin{itemize}
        \item Inspired by the Oxford University branding guidelines
        \item Uses the \texttt{Copenhagen} theme as its base
        \item Rich in features and customizations
    \end{itemize}
\end{frame}

\begin{frame}{Introduction (cont'd)}
    This is a sample slide in the \textbf{Introduction} section.

    \begin{equation}
        \begin{split}
            \frac{\partial \rho}{\partial t} + \nabla \cdot (\rho \pmb{u}) &= 0 \\
            \frac{\partial \rho \pmb{u}}{\partial t} + \nabla \cdot (\rho \pmb{u} \pmb{u}^T + p \pmb{I}) &= \pmb{f} \\
            \frac{\partial \rho E}{\partial t} + \nabla \cdot (\rho E \pmb{u} + p \pmb{u}) &= \pmb{u} \cdot \pmb{f}
        \end{split}
    \end{equation}

\end{frame}

\section{Custom Features}

\begin{frame}[fragile]{Code Example}
Here's an example of highlighted code using \texttt{minted}:

\begin{minted}{python}
def fibonacci(n):
    """Return the first n numbers of the Fibonacci sequence."""
    if n <= 0:
        return []
    elif n == 1:
        return [0]
    elif n == 2:
        return [0, 1]
    else:
        fib_sequence = [0, 1]
        while len(fib_sequence) < n:
            next_num = fib_sequence[-1] + fib_sequence[-2]
            fib_sequence.append(next_num)
        return fib_sequence
\end{minted}
\end{frame}

\begin{frame}{Block Examples}
\begin{block}{Normal Block Title}
This is a normal block.
\end{block}

\begin{alertblock}{Alerted Block Title}
This is an alerted block.
\end{alertblock}

\begin{exampleblock}{Example Block Title}
This is an example block.
\end{exampleblock}
\end{frame}

\backmatter

\end{document}
